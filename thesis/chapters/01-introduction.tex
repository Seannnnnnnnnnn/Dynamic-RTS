\clearpage

\def\chaptertitle{Introduction}

\lhead{\emph{\chaptertitle}}

\chapter{\chaptertitle}
\label{ch:introduction}

This thesis studies an applied problem from Computational Geometry called \textit{Range Thresholding on Streams} \cite{GAN16}. The problem is best intuited through an example related to the share market. Traders may pay close attention to the trading volumes in sensitive price ranges as for example. substantial limit orders in certain price ranges may indicate significant price downfall. A trader therefore may wish to receive timely alerts for queries of the form: 

\textit{"Alert me when \text{100,000} shares of AAPL (Apple Inc.) have been sold in the price range \text{[200, 205]} from now."}

More formally, the Range Threshold on Streams (RTS) problem can be formulated as follows. Consider a sequence of elements $\{e_i\}^{n}_{i=1}$ where each element $e_i$ contains a \textit{value} $v(e_i) \in \mathbb{R}$ and a \textit{weight} $w(e_i) \in \mathbb{Z}$. An RTS query $q$ comprises of a tuple $(R_q, \tau_q)$ where $R_q = [x,y]$ is a subset of the real line, and $\tau_q \in\mathbb{N}$ is some \textit{threshold}

The challenge of this thesis is to design an efficient algorithm / data structure that supports a large number of such queries. 

\section{Problem Statement}
\label{sec:problem-statement}




Currently, the Wybe language supports only first-order programming, where all terms have first-order types. In this thesis, we introduce an extension to the Wybe language that supports higher-order programming.

While many languages support higher-order programming, the Wybe language supports features that, to our knowledge, have not been investigated in a higher-order context. One such novel language feature Wybe has is a resource system. Resources in Wybe are akin to global variables in imperative languages or class variables in object-oriented languages. Resources, however, are more constrained in their usage, with each procedure that manipulates a resource requiring being marked as such. 

The resource system has not been investigated in the context of higher-order programming, providing a space in which we can investigate this language feature in a higher-order context. The desired semantics motivate an extension to the intermediate representation that introduces global variables. We aim to extend the Wybe language with our desired semantics of resources and provide novel optimisations enabled by these semantics. 

With these extensions to the Wybe language, we evaluate the performance of the language in terms of execution time and program size, with the existing language as a baseline. While a slow-down may seem detrimental to the utility of a language, higher-order programming increases the expressiveness of a language. The increased expressiveness allows more general programs to be written with less source code and programming effort. Ideally, these overheads should be relatively small in comparison to the overall runtime of a program, lowering the cost of these extensions.

Hence, we aim to answer these research questions:
\begin{enumerate}
  \item How can the Wybe language be extended with higher-order programming?
  \item How can the resource system of Wybe be extended to support higher-order programming while maintaining the guarantees of resources and the expressiveness of higher-order programming?
  \item Do these extensions perform similarly in terms of execution runtime and program size when compared with the existing Wybe language?
\end{enumerate}

\section{Thesis Outline}
\label{sec:doc-outline}

The remainder of this chapter continues with an overview of the Wybe language. Following, in \cref{ch:preliminaries}, we review literature regarding the design an implementation of higher-order programming, type systems, and intermediate representations.
