\clearpage

\def\chaptertitle{Introduction}

\lhead{\emph{\chaptertitle}}

\chapter{\chaptertitle}
\label{ch:introduction}

This thesis studies an applied problem from Computational Geometry called \textit{Range Thresholding on Streams} \cite{GAN16}. The problem is best motivated through an example related to trading in the share market. High frequency traders often pay close attention to the trading volumes in sensitive price ranges. For example, traders may know that substantial limit orders in certain price ranges may indicate significant price downfall. We therefore wish to receive timely alerts for queries of the form: 

\textit{"Alert me when \text{100,000} shares of AAPL (Apple Inc.) have been sold in the price range \text{[200, 205]} from now."}

Or more generally, we may even wish to receive an alert for a query involving multiple trading streams, such as 

\textit{"Alert me when \text{100,000} share of APPL have been solve in the price range of [200, 205] and the NASDAQ is trading between [4,000, 4,500]"}

Both queries are example of so-called \textit{Range Thresholding Queries}. Designing efficient algorithms to simulataneoulsy support massive amounts of such queries is central to the Range Thresholding on Streams problem. 

In mathematical language, the Range Threshold on Streams (RTS) problem can be formulated as follows. Consider an unbounded stream of elements $\{e_i\}^{n}_{i=1}$ where each element $e_i$ contains a \textit{value} $v(e_i) \in \mathbb{R}$ and a \textit{weight} $w(e_i) \in \mathbb{Z}$. An RTS query $q$ comprises of a tuple $(R_q, \tau_q)$ where $R_q = [x,y]$ is a subset of the real line, and $\tau_q \in\mathbb{N}$ is some \textit{threshold}. Let $S(q)$ be the set of stream elements $e_i$ such that 
\begin{itemize}
    \item $e_i$ is received after the time $q$ is registered and,
    \item has value $v(e)$ falling into $R_q$. 
\end{itemize}
The query \textit{matures} the instant that $\sum_{e\in S(q)}w(e)\geq \tau_q$. The Range Thresholding on Streams problem is to design an algorithm or data structure that can efficiently support and process a huge number of simultaneous RTS queries  in terms of both space and computational cost. 

As described in the RTS literature \cite{GAN16, DBLP:conf/sigmod/ZhangGBKCZ22}, an RTS query $q$ intuitively detects when the interval $R_q$ becomes a \textit{hot spot}, and solutions to the RTS problem are required for applications which involve rapid responses to such hot spots, such as the as our earlier example in stock trading.

The RTS problem is deceptively simple at first glance - one can easily solve the problem with the following polynomial-time algorithm; whenever a stream element $e$ arrives, one checks if $e\in R_q$ for each registered query $q$, and update some counter $c_q$ if so. The instant that $c_q \geq \tau_q$ report maturity for the query. For a set of $m$ queries the cost of processing a $n$ streams becomes a prohibitively large $O(nm)$ - suggesting that the naive algorithm becomes computationally intractable when attempting to support a large number of queries. 

Despite the number of applications for the RTS problem \cite{DBLP:conf/sigmod/ZhangGBKCZ22}, it was only until recently that the first algorithm was proposed that escapes the $O(nm)$ barrier of the naive algorithm \cite{GAN16}. 

\section{Motivation}
\label{sec:motivation}
This thesis is primarily motivated to extend the literature to define and solve a more general version of the RTS problem, in which the interval of interest for a given query $R_q$, is allowed to become \textit{dynamic} and \textit{move} with the evolution of the data stream. 

For a motivating example of this, we return to the context of High Frequency Trading. Traders will often be interested in buy/sell orders being placed within ranges related to the so-called \textit{volume weighted average price} (VWAP) \cite{vwap}. Recall that the formula for VWAP is given by 
\begin{equation}
    \frac{\sum_{e}v(e) \times w(e)}{\sum_{e}w(e)}
\end{equation}
where $v(e)$ denotes the \textit{price} or \textit{value} of order $e$ and $w(e)$ denotes the corresponding \textit{volume} or \textit{weight}. If a trader were to register an alert of such as:

\textit{"Notify me when $\tau$ shares of limit orders have been registered within $x$\% of VWAP"}

As we can see in (1), the endpoints of such a query will change as new stream elements are processed - with the potential for particularly large jumps for trades with large volumes. Existing algorithms to solve the RTS problem are also unable to support such queries, as they assume that the registered interval $R_q$ remains unaltered over the lifetime of the query. 

Our goal for this thesis is to mathematically formalise this more general type of query, where the endpoints or region of interest $R_q$ is allowed to \textit{move} with the introduction of each new stream element. Similar to the RTS problem, we then aim to design scalable algorithms to manage a large number of simultaneous queries. 

\section{Thesis Outline}
\label{sec:doc-outline}

The remainder of the thesis is outlined as follows; in \cref{ch:preliminaries} we cover all the necessary mathematical material to understand and analyse the state of the art solutions to the RTS problem. In \cref{ch:rts} we formally define the RTS problem, and briefly cover some existing solutions to RTS. The bulk of the chapter is then devoted to describing the state of the art \textit{DT Algorithm} for the RTS problem, and formally proving it's correctness and sub-quadratic run time. In \cref{ch:drts} we formalise an extension of the problem, which we call \textit{Dynamic Range Thresholding on Streams} (DRTS), as described in \cref{sec:motivation}. For the remainder of the chapter, we demonstrate how the DT Algorithm can be extended to solve the dynamic version of the problem, as well as propose a novel algorithm, in which we trade accuracy for speed to solve a special class of DRTS instances. Finally, we demonstrate that our algorithms not only have strong theoretical properties, but also perform well in practice via an experimental evaluation on synthetic and real data. 
